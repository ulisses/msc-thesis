\chapter*{Abstract}
ESA (European Space Agency) uses a big engine to perform tests in the
Ground Segment infrastructure, specially the Operational Simulator.
This engine uses many different tools to ensure the development of
regression testing infrastructure and these tests perform black-box
testing to the C++ simulator implementation.
VST (Vision Space Technologies) is one of the companies that provides
these services to ESA and they need a tool to instead of writing
manually scripts to perform tests, it should try to automatically
infer tests from the existing C++ code.
There are many approaches trying to tackle this problem and therefore
many tools were developed. Here we will present a study on the most
recent tools that uses: Model-based testing, Specification-based
testing, Constraint-based generation, Random generation and
Grammar-based generation for the most used languages - C, JAVA and C\#.

Automated Test Case Generation tools give support for creating test
cases and at the same time ensure test case coverage methodically. The
main goal of this tools is extract information from the program on how
to generate executable test cases.
Using manual written tests is tedious, time consuming and error-prone.
Lots of functions/methods need full code coverage and this technique
leaves to incomplete test suites and is hard to create tests that
cover specific code paths potentially leaving many hidden bugs.
Besides that, software is not a static artifact and is constantly
evolving, so a test generation technique could be a more suitable
mechanism in the development process.
We will be mainly focused on the white-box automatic test generation techniques.

Since no other testing tool used by ESA makes any use of a formal
model we decide to start by infer the UML model from the existing C++
code and try to explore the generation of Object Constraint Language specifications
from C++ code and UML diagrams. The idea is to capture the requirements properties about the code, and from this beef up the white-box automatic test generation.

%\addtocounter{page}{-1}
