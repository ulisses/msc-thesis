\documentclass{beamer}

\mode<presentation>
{
   \usetheme{EEng}
  \setbeamercovered{transparent}
  \setbeamercolor{background canvas}{bg=black!0}
}

\usepackage{movie15}
\usepackage{enumerate}
\usepackage{array}
\usepackage{graphics}
\usepackage{ucs}
\usepackage[utf8x]{inputenc}
\usepackage[english]{babel}
\usepackage{amsmath, amsthm, amssymb}
\usepackage{amsmath}
\usepackage{amsfonts}
\usepackage{xcolor}
\usepackage{url}

\title{Automatic Test Generation for Space}
\author{Ulisses Costa}
\date{\today}

\begin{document}
\begin{frame}
   \titlepage
\end{frame}

\begin{frame}\frametitle{Contexto}
\begin{block}{Problema}
A VST (Visionspace Technologies) presta serviços relacionados com teste para a ESA (Agência Espacial Europeia) e pretende automatizar
a parte de geração de testes para a plataforma em que trabalha, o Operational Simulator da Ground Segment Infrastructure.
\end{block}{}

Esta dissertação tem como objectivos:
\begin{itemize}
\item Gerar testes de forma automática para Operational Simulator
\item Conseguir gerar testes unitários para a linguagem em que o Operational Simulator está escrito, C++
\item Parametrizar o tamanho das estruturas de dados geradas e outros atributos
\end{itemize}
\end{frame}

\begin{frame}\frametitle{Motivação}
Proposta:
\begin{itemize}
\item Extrair UML e OCL apartir do código
\item Extrair testes apartir do código
\end{itemize}
\end{frame}

\begin{frame}\frametitle{White vs. Black Box Testing}
\begin{block}{Tipos de testes relativamente ao conhcimento do código}
\begin{description}
\item[White Box], há conhecimento do código fonte e este é usado para a tarefa de teste.
\item[Black Box], há conhecimento apenas dos requisitos funcionais do código e de como cada componente do código deverá comportar-se.
\end{description}
\end{block}
%\begin{block}{Técnicas de White Box}
%\begin{description}
%\item[Symbolic execution],
%\item[], 
%\end{itemize}
%\end{block}
\end{frame}

\begin{frame}\frametitle{Abordagens}
\begin{description}
\item[Specification-based Generation Testing], aka Model Based Testing consiste em testar um programa baseando-se na especificação ou no modelo do programa.
Em especifico podem ser gerados testcases baseados na especificação do comportamento do programa, sem ser necessário a implementação do programa.
\item[Constraint-based Generation Testing], pode ser usado para seleccionar testcases que satisfaçam determinadas restrições sobre um conjunto de variaveis.
Pode ser combinado com symbolic execution, portanto o programa pode ser executado simbolicamente, colectando restrições ao longo dos diferentes caminhos no CFG
e posteriormente resolver estas restrições e produzir testcases apartir daí.
\end{description}
\end{frame}

\begin{frame}\frametitle{Estado actual}
Até agora foram estudadas várias ferramentas, as mais importantes e mais desenvolvidas:
\begin{itemize}
\item Korat, é uma framework madura paar construir automaticamente estruturas complexas em JAVA
\item Pex, é uma ferramenta automatica da Microsoft de White-box testing que garante full coverage
\end{itemize}

\begin{block}{Sumário}
Pex usa execução estática e é muito eficiente em descobrir todos os feasible paths em métodos C\#.
Pex tambem foi utilizado para gerar testcases das classes, mas as instâncias geradas não mantinham os invariantes das estruturas de dados.\\

Por outro lado o Korat é a ferramenta ideal para a tarefa de gerar estruturas que cumpram os invariantes pedidos.
\end{block}
\end{frame}

\end{document}

